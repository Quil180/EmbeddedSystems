\documentclass{article}
\usepackage{graphicx} % Required for inserting images
\usepackage{varwidth}
\usepackage{xcolor}
\usepackage{listings}


\definecolor{codegreen}{rgb}{0,0.6,0}
\definecolor{codegray}{rgb}{0.5,0.5,0.5}
\definecolor{codepurple}{rgb}{0.58,0,0.82}
\definecolor{backcolour}{rgb}{0.95,0.95,0.92}

\lstdefinestyle{CStyle}{
	language=C++,
	backgroundcolor=\color{backcolour},   
	commentstyle=\color{codegreen},
	keywordstyle=\color{magenta},
	numberstyle=\tiny\color{codegray},
	stringstyle=\color{codepurple},
	basicstyle=\ttfamily\footnotesize,
	breakatwhitespace=false,         
	breaklines=true,                 
	keepspaces=true,                 
	numbers=left,       
	numbersep=5pt,                  
	showspaces=false,                
	showstringspaces=false,
	showtabs=false,                  
	tabsize=2,
}
\lstset{style=CStyle}

\title{Lab 10 Report \\ \large EEL4742C - 00446}
\author{Yousef Awad}
\date{November 2025}
\setcounter{secnumdepth}{0}

\begin{document}

\maketitle
\tableofcontents
\newpage

\section{Introduction}
This experiment investigates the advanced capabilities of the MSP430FR6989 Timer\_A module, focusing on using multiple channels to handle concurrent hardware events. Objectives include implementing simultaneous timing intervals to control independent LED flashing rates in continuous mode, generating Pulse Width Modulation (PWM) signals to adjust LED brightness without CPU intervention , and utilizing Input Capture mode to timestamp external user input.

\section{10.1 Timer's Multiple Channels}
The visual comparison appears to be the same/similar to what I set the values to be at. Alongside this here is the following derivation for the number of cycles for each channel:
$$ \frac{f_{clk}}{input\ divider}*desired\ delay\ in\ seconds $$
$$ \frac{32768}{4}*0.1 \approx 819 \rightarrow TA0CCR0 = 819 $$
$$ \frac{32768}{4}*0.5 \approx 4096 \rightarrow TA0CCR1 = 4096 $$
\lstinputlisting{10_1.c}
\pagebreak

\section{10.2 Using Three Channels}
The durations, at least when tested by my stopwatch on my phone and on my watch, are close to the time's that I set (though, there was a slight error most likely due to human error).
\lstinputlisting{10_2.c}
\pagebreak

\section{10.3 Driving a PWM Signal on the Pin}
The brightness level, thankfully, is changed via the changing of TA0CCR1 in all the ranges.
\lstinputlisting{10_3.c}
\pagebreak

\section{10.4 Timer Input Capture}
The button push lasts \textbf{[answer here]}
\lstinputlisting{10_4.c}
\pagebreak

\section{Student Q\&A}
\subsection{1}
\textbf{Given:} \textit{Copy the description of P9.7 from the pinout diagram and determine whether this pin supports timer-based output.}
\newline
From the description of P9.7 is has the following functionality ESICI3/A15/C15. In that, none of these support CCR, therefore the pin does not support timer-based output.

\subsection{2}
\textbf{Given:} \textit{In the code with three channels, why was it necessary to divide ACLK?}
\newline
It is necessary to divide ACLK due to the fact that we want a delay of 4 seconds. ACLK specifically has a frequency of 65535, of which in up mode (and/or continuous mode) will have a maximum period of two seconds between each interrupt. Therefore, if we divide it, we can extend the interrupts to be at most every four seconds.

\subsection{3}
\textbf{Given:} \textit{In the first part, we configured two periodic interrupts using two channels of the timer. Is this
approach scalable? For example, using a Timer A module with five channels, can we configure five periodic interrupts? Explain and mention in what mode the timer would run.}
\newline
This approach is scalable up to five channels, since Timer\_A has support up to five channels. The other three modules of Timer only support two, three, and three channels respectively. If you wanted to have these five channels, the timer would then have to be configured with a mode that allows for the capture and compare registers to be utilized (eg: up mode). Continuous mode would, specifically, not allow for the unique interrupts to be raised.

\subsection{4}
\textbf{Given:} \textit{As an example, Channel 1’s interrupt occurs every 40K cycles. The first interrupt is scheduled for when TAR=40K cycles. Explain how the next interrupt is scheduled? Explain the overflow mechanism and show why it results in a correct value.}
\newline
With interrupts occuring every 40 thousand cycles, future interrupts are scheduled by adding the interrupt period to the TA0R value of the previous interrupt. In this case, it would be 40 thousand + 40 thousand, resulting in 80 thousand. This 80 thousand is greater than the maximum therefore making so that, when it overflows, you get 0x3380 in hex or 14464 in decimal, of which is now the new expected TA0R value for the next interrupt raise.

\section{Conclusion}
This laboratory successfully demonstrated the advanced, hardware-centric features of the Timer\_A module, moving beyond simple software delays. We proved that multiple capture/compare channels can operate concurrently in continuous mode to manage independent, asynchronous timing intervals. Furthermore, this lab highlighted the efficiency of hardware offloading by successfully generating a stable, CPU-independent 1000Hz PWM signal to control LED brightness and by utilizing Input Capture mode to timestamp external button events with high precision. These exercises confirm that leveraging the timer's dedicated hardware for output modulation and input timing is essential for building efficient, responsive, and low-power embedded systems.

\end{document}
