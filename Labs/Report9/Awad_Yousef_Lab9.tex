\documentclass{article}
\usepackage{graphicx} % Required for inserting images
\usepackage{varwidth}
\usepackage{xcolor}
\usepackage{listings}


\definecolor{codegreen}{rgb}{0,0.6,0}
\definecolor{codegray}{rgb}{0.5,0.5,0.5}
\definecolor{codepurple}{rgb}{0.58,0,0.82}
\definecolor{backcolour}{rgb}{0.95,0.95,0.92}

\lstdefinestyle{CStyle}{
	language=C++,
	backgroundcolor=\color{backcolour},   
	commentstyle=\color{codegreen},
	keywordstyle=\color{magenta},
	numberstyle=\tiny\color{codegray},
	stringstyle=\color{codepurple},
	basicstyle=\ttfamily\footnotesize,
	breakatwhitespace=false,         
	breaklines=true,                 
	keepspaces=true,                 
	numbers=left,       
	numbersep=5pt,                  
	showspaces=false,                
	showstringspaces=false,
	showtabs=false,                  
	tabsize=2,
}
\lstset{style=CStyle}

\title{Lab 9 Report \\ \large EEL4742C - 00446}
\author{Yousef Awad}
\date{November 2025}
\setcounter{secnumdepth}{0}

\begin{document}

\maketitle
\tableofcontents
\newpage

\section{Introduction}

\section{9.1 Using the ADC SAR-Type}
Now, for this setup I used SPI Mode 1.
\lstinputlisting{9_1.c}
\pagebreak

\section{9.2 Reading the X- and Y- Coordinates of the Joystick}
\includegraphics[width=1\textwidth]{pictures/demo_showing_2_pictures.png}
\lstinputlisting{9_2.c}
\pagebreak

\section{9.3 Application: Platform Balancing Control}
\lstinputlisting{9_3.c}
\pagebreak

\section{Student Q\&A}
\subsection{1}
\textbf{Given:} \textit{Is SPI implemented as simplex or full-duplex in this experiment?}
SPI is only implmented as simplex as the Serial Data Out Pin does not transmit data back to the MCU. This means that there is only one lane of communication, thereby not being duplexed.

\subsection{2}
\textbf{Given:} \textit{What SPI clock frequency did we set up in this lab?}
The SPI Clock Frequency was set to 8MHz as instructed in the lab manual...

\subsection{3}
\textbf{Given:} \textit{What I2C clock frequency did we set up in this lab?}
The I2C Clock Frequency was set to 320KHz as was instructed by the lab manual....

\subsection{4}
\textbf{Given:} \textit{What is the maximum SPI clock frequency that is supported by the eUSCI module? Look in the microcontroller’s data sheet in Table 5-18.}
The Maximum SPI Clock Frequency supported by the eUSCI Module for the MSP430 is 16MHz, according to the data sheet (Table 5-18, shown below).
\newline
\includegraphics[width=1\textwidth]{pictures/image.png}

\subsection{5}
\textbf{Given:} \textit{Show how you computed the I2C clock divider in the 3rd part.}
To calculate the I2C Clock Divider for Part 3, I simply did teh following:
$$ Clock = \frac{16MHz_{master\ clock}}{320KHz_{desired\ clock}} = \frac{16*10^6}{320*10^3} = 50 $$

\section{Conclusion}

\end{document}
