\documentclass{article}

\usepackage{fancyhdr}
\usepackage{extramarks}
\usepackage{amsmath}
\usepackage{amsthm}
\usepackage{amsfonts}
\usepackage{tikz}
\usepackage[plain]{algorithm}
\usepackage{algpseudocode}
\usepackage{xcolor}
\usepackage{listings}

\definecolor{codegreen}{rgb}{0,0.6,0}
\definecolor{codegray}{rgb}{0.5,0.5,0.5}
\definecolor{codepurple}{rgb}{0.58,0,0.82}
\definecolor{backcolour}{rgb}{0.95,0.95,0.92}

\lstdefinestyle{CStyle}{
	language=C++,
	backgroundcolor=\color{backcolour},   
	commentstyle=\color{codegreen},
	keywordstyle=\color{magenta},
	numberstyle=\tiny\color{codegray},
	stringstyle=\color{codepurple},
	basicstyle=\ttfamily\footnotesize,
	breakatwhitespace=false,         
	breaklines=true,                 
	keepspaces=true,                 
	numbers=left,       
	numbersep=5pt,                  
	showspaces=false,                
	showstringspaces=false,
	showtabs=false,                  
	tabsize=2,
}
\lstset{style=CStyle}


\usetikzlibrary{automata,positioning}

%
% Basic Document Settings
%

\topmargin=-0.45in
\evensidemargin=0in
\oddsidemargin=0in
\textwidth=6.5in
\textheight=9.0in
\headsep=0.25in

\linespread{1.1}

\pagestyle{fancy}
\lhead{Yousef Alaa Awad}
\chead{\hmwkClass\: \hmwkTitle}
\rhead{\firstxmark}
\lfoot{\lastxmark}
\cfoot{\thepage}

\renewcommand\headrulewidth{0.4pt}
\renewcommand\footrulewidth{0.4pt}

\setlength\parindent{0pt}

%
% Create Problem Sections
%

\setcounter{secnumdepth}{0}
\newcounter{partCounter}
\newcounter{homeworkProblemCounter}
\setcounter{homeworkProblemCounter}{1}

\newcommand{\hmwkTitle}{Homework\ \#1}
\newcommand{\hmwkDueDate}{September 14, 2025}
\newcommand{\hmwkClass}{Embedded Systems}

%
% Title Page
%

\title{
    \vspace{2in}
    \textmd{\textbf{\hmwkClass:\ \hmwkTitle}}\\
    \normalsize\vspace{0.1in}
    \vspace{3in}
}

\author{Yousef Alaa Awad}

% Problems start here
\begin{document}

\maketitle
\pagebreak

\section{1}
\textbf{Given:} For the questions below, write the code using the masks that are pre-defined in the header file. Perform the operations below on the 8-bit variable (uint\_8t data).

\subsection{A) Write code that independently sets bit5, clears bit5, and inverts bit5}
\begin{lstlisting}
data |= BIT5;  // Ensuring BIT5 is on.
data &= ~BIT5; // Setting BIT5 to 0, but leaving the rest the same
data ^= BIT5;  // Inverting only BIT5.
\end{lstlisting}

\subsection{B) Write code that independently sets bit2 and bit3, clears bit2 and bit3, inverts bit2 and bit3, and sets bit2 and clears bit3}
\begin{lstlisting}
data |= (BIT2 | BIT3);        // Setting bit2 and bit3
data &= ~(BIT2 | BIT3);       // Clearing bit2 and bit3
data ^= (BIT2 | BIT3);        // Inverting bit2 and bit3
data = (data | BIT2) & ~BIT3; // Setting bit2 and clearing bit3
\end{lstlisting}

\subsection{C) Write an if-condition line that independently checks if bit4 is 1, checks if bit4 is 0, checks if bits4 and 5 are both 1, checks if bit4 is 0 and bit5 is 1, and checks if bits 4 and 5 are both 0}
The following are just snippets
\begin{lstlisting}
if (!(data & BIT4)) // Checks if bit4 is 1 by only leaving bit4 via the &

if (data & BIT4) // Checks if bit4 is 0

if (!(data & (BIT4 | BIT5))) // Checks if bit4 and bit5 are 1

if ((data & BIT4) && !(data & BIT5)) // Checks if bit4 is 0 and bit5 is 1

if (data & (BIT4 | BIT5)) // Checks if bit4 and bit5 are both 0
\end{lstlisting}

\section{2}
\textbf{Given: } A module on the microcontroller is configured using a control register called CTL that haas the format shown below.
\begin{table}[h!]
    \centering
    \begin{tabular}{|c|c|c|c|}
        \hline
        SLP & CLK & CAP & IE \\
        2 bits & 3 bits & 2 bits & 1 bit \\
        \hline
    \end{tabular}
\end{table}
\begin{itemize}
	\item SLP: Selects sleep mode (0-3)
	\item CLK: Selects clock speed (0-7)
	\item CAP: Selects built-in capacitor value (0-3)
	\item IE: Interrupt enable bit (0/1)
\end{itemize}

\subsection{A) Write a line that configures: Sleep mode 3, clock speed 4, capacitor value 1, and interrupts enabled}
\begin{lstlisting}
CTL = (SLP_3 |  CLK_4 | CAP_1 | IE);
\end{lstlisting}

\subsection{B) Using Part A, show the masks used and the final value of CTL in binary}
The following are the original masks:
\begin{itemize}
	\item SLP\_3 = 1100 0000
	\item CLK\_4 = 0010 0000
	\item CAP\_1 = 0000 0010
	\item IE\ \ \ \ \ \ \ = 0000 0001
\end{itemize}
Therefore, the total mask will look like the following
\begin{table}[h!]
    \centering
    \begin{tabular}{|c|c|c|c|}
        \hline
				11 & 100 & 01 & 1\\
        \hline
    \end{tabular}
\end{table}
\newline
or \textbf{1110 0011}.

\subsection{C) Write a piece of code that changes SLP to 1 from any unknown value of SLP}
\begin{lstlisting}
CLT = (CTL & ~SLP_1) | SLP_1;
\end{lstlisting}

\subsection{D) Write an if-condition that checks if SLP is 3}
\begin{lstlisting}
if ((CTL & SLP_3) == SLP_3)
\end{lstlisting}

\subsection{E) Write an if-condition that checks if the current value of CLK is 0, 2, 4, or 6}
\begin{lstlisting}
if ((CTL & CLK_1) != CLK_1)
\end{lstlisting}

\section{3}
\subsection{A) A memory is byte addressable and has an 18-bit address. All the addresses are value, what is the total size of the memory?}
First, with a 18-bit long address bus, we know that there are $2^18$ possible addresses. Alongside this, since each address is simply 1 byte, we know that there are $2^18$ bytes. Therefore we have the following math...
$$ 2^{18}\ \text{addresses} = 2^{18}\ \text{bytes} = 2^{10}*2^8\ \text{bytes} = 2^8\ \text{kilobytes} = 256\ \text{kilobytes}$$

\subsection{B) A memory is byte addressable and has a total size of 17,408 bytes (17KB). What is the smallest address size that can be used for this memory?}
With 17KB of memory, you have $2^4+1$ kilobytes, of which is $2^{10}*(2^4 + 1)$ bytes of which is $2^{14}+2^{10}$ bytes of which means we have 14 whole bits of space, alongside another full 1024 addresses needed. Meaning we need \textbf{15} bits of an address bus to accomodate 17KB of memory.

\section{4}
\subsection{A) A microcontroller's memory map allocates the FLASH code space to the address range 0x0500 to 0x0CFF. What is the code size, in bytes, that is supported by this microcontroller?}
$$ 0xCFF - 0x500 = 0x7FF \rightarrow 0x7FF = 2047 \rightarrow 2047 + 1\text{ (for base)} = 2048\text{ Bytes} = 2\text{KB} $$

\subsection{B) The vector table contains memory addresses (a vector is memory address). In a certain MSP430 device, the vector table is in the range 0xFFC0 to 0cFFFF. The memory address is 16-bit. How many vectors does this vector table support?}
\begin{itemize}
	\item $ 0xFFFF-0xFFC0 = 0x003F $
 	\item $ 0xx003F = 16^1*3 + 16^0*15 = 63 $
	\item $ 16 + 1\text{ (for base)} = 64$
	\item $ \frac{64}{2}\text{ 2 bytes per vector} = \textbf{32 vectors}$ 
\end{itemize}

\end{document}
